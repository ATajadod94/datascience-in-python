
% Default to the notebook output style

    


% Inherit from the specified cell style.




    
\documentclass[11pt]{article}

    
    
    \usepackage[T1]{fontenc}
    % Nicer default font (+ math font) than Computer Modern for most use cases
    \usepackage{mathpazo}

    % Basic figure setup, for now with no caption control since it's done
    % automatically by Pandoc (which extracts ![](path) syntax from Markdown).
    \usepackage{graphicx}
    % We will generate all images so they have a width \maxwidth. This means
    % that they will get their normal width if they fit onto the page, but
    % are scaled down if they would overflow the margins.
    \makeatletter
    \def\maxwidth{\ifdim\Gin@nat@width>\linewidth\linewidth
    \else\Gin@nat@width\fi}
    \makeatother
    \let\Oldincludegraphics\includegraphics
    % Set max figure width to be 80% of text width, for now hardcoded.
    \renewcommand{\includegraphics}[1]{\Oldincludegraphics[width=.8\maxwidth]{#1}}
    % Ensure that by default, figures have no caption (until we provide a
    % proper Figure object with a Caption API and a way to capture that
    % in the conversion process - todo).
    \usepackage{caption}
    \DeclareCaptionLabelFormat{nolabel}{}
    \captionsetup{labelformat=nolabel}

    \usepackage{adjustbox} % Used to constrain images to a maximum size 
    \usepackage{xcolor} % Allow colors to be defined
    \usepackage{enumerate} % Needed for markdown enumerations to work
    \usepackage{geometry} % Used to adjust the document margins
    \usepackage{amsmath} % Equations
    \usepackage{amssymb} % Equations
    \usepackage{textcomp} % defines textquotesingle
    % Hack from http://tex.stackexchange.com/a/47451/13684:
    \AtBeginDocument{%
        \def\PYZsq{\textquotesingle}% Upright quotes in Pygmentized code
    }
    \usepackage{upquote} % Upright quotes for verbatim code
    \usepackage{eurosym} % defines \euro
    \usepackage[mathletters]{ucs} % Extended unicode (utf-8) support
    \usepackage[utf8x]{inputenc} % Allow utf-8 characters in the tex document
    \usepackage{fancyvrb} % verbatim replacement that allows latex
    \usepackage{grffile} % extends the file name processing of package graphics 
                         % to support a larger range 
    % The hyperref package gives us a pdf with properly built
    % internal navigation ('pdf bookmarks' for the table of contents,
    % internal cross-reference links, web links for URLs, etc.)
    \usepackage{hyperref}
    \usepackage{longtable} % longtable support required by pandoc >1.10
    \usepackage{booktabs}  % table support for pandoc > 1.12.2
    \usepackage[inline]{enumitem} % IRkernel/repr support (it uses the enumerate* environment)
    \usepackage[normalem]{ulem} % ulem is needed to support strikethroughs (\sout)
                                % normalem makes italics be italics, not underlines
    

    
    
    % Colors for the hyperref package
    \definecolor{urlcolor}{rgb}{0,.145,.698}
    \definecolor{linkcolor}{rgb}{.71,0.21,0.01}
    \definecolor{citecolor}{rgb}{.12,.54,.11}

    % ANSI colors
    \definecolor{ansi-black}{HTML}{3E424D}
    \definecolor{ansi-black-intense}{HTML}{282C36}
    \definecolor{ansi-red}{HTML}{E75C58}
    \definecolor{ansi-red-intense}{HTML}{B22B31}
    \definecolor{ansi-green}{HTML}{00A250}
    \definecolor{ansi-green-intense}{HTML}{007427}
    \definecolor{ansi-yellow}{HTML}{DDB62B}
    \definecolor{ansi-yellow-intense}{HTML}{B27D12}
    \definecolor{ansi-blue}{HTML}{208FFB}
    \definecolor{ansi-blue-intense}{HTML}{0065CA}
    \definecolor{ansi-magenta}{HTML}{D160C4}
    \definecolor{ansi-magenta-intense}{HTML}{A03196}
    \definecolor{ansi-cyan}{HTML}{60C6C8}
    \definecolor{ansi-cyan-intense}{HTML}{258F8F}
    \definecolor{ansi-white}{HTML}{C5C1B4}
    \definecolor{ansi-white-intense}{HTML}{A1A6B2}

    % commands and environments needed by pandoc snippets
    % extracted from the output of `pandoc -s`
    \providecommand{\tightlist}{%
      \setlength{\itemsep}{0pt}\setlength{\parskip}{0pt}}
    \DefineVerbatimEnvironment{Highlighting}{Verbatim}{commandchars=\\\{\}}
    % Add ',fontsize=\small' for more characters per line
    \newenvironment{Shaded}{}{}
    \newcommand{\KeywordTok}[1]{\textcolor[rgb]{0.00,0.44,0.13}{\textbf{{#1}}}}
    \newcommand{\DataTypeTok}[1]{\textcolor[rgb]{0.56,0.13,0.00}{{#1}}}
    \newcommand{\DecValTok}[1]{\textcolor[rgb]{0.25,0.63,0.44}{{#1}}}
    \newcommand{\BaseNTok}[1]{\textcolor[rgb]{0.25,0.63,0.44}{{#1}}}
    \newcommand{\FloatTok}[1]{\textcolor[rgb]{0.25,0.63,0.44}{{#1}}}
    \newcommand{\CharTok}[1]{\textcolor[rgb]{0.25,0.44,0.63}{{#1}}}
    \newcommand{\StringTok}[1]{\textcolor[rgb]{0.25,0.44,0.63}{{#1}}}
    \newcommand{\CommentTok}[1]{\textcolor[rgb]{0.38,0.63,0.69}{\textit{{#1}}}}
    \newcommand{\OtherTok}[1]{\textcolor[rgb]{0.00,0.44,0.13}{{#1}}}
    \newcommand{\AlertTok}[1]{\textcolor[rgb]{1.00,0.00,0.00}{\textbf{{#1}}}}
    \newcommand{\FunctionTok}[1]{\textcolor[rgb]{0.02,0.16,0.49}{{#1}}}
    \newcommand{\RegionMarkerTok}[1]{{#1}}
    \newcommand{\ErrorTok}[1]{\textcolor[rgb]{1.00,0.00,0.00}{\textbf{{#1}}}}
    \newcommand{\NormalTok}[1]{{#1}}
    
    % Additional commands for more recent versions of Pandoc
    \newcommand{\ConstantTok}[1]{\textcolor[rgb]{0.53,0.00,0.00}{{#1}}}
    \newcommand{\SpecialCharTok}[1]{\textcolor[rgb]{0.25,0.44,0.63}{{#1}}}
    \newcommand{\VerbatimStringTok}[1]{\textcolor[rgb]{0.25,0.44,0.63}{{#1}}}
    \newcommand{\SpecialStringTok}[1]{\textcolor[rgb]{0.73,0.40,0.53}{{#1}}}
    \newcommand{\ImportTok}[1]{{#1}}
    \newcommand{\DocumentationTok}[1]{\textcolor[rgb]{0.73,0.13,0.13}{\textit{{#1}}}}
    \newcommand{\AnnotationTok}[1]{\textcolor[rgb]{0.38,0.63,0.69}{\textbf{\textit{{#1}}}}}
    \newcommand{\CommentVarTok}[1]{\textcolor[rgb]{0.38,0.63,0.69}{\textbf{\textit{{#1}}}}}
    \newcommand{\VariableTok}[1]{\textcolor[rgb]{0.10,0.09,0.49}{{#1}}}
    \newcommand{\ControlFlowTok}[1]{\textcolor[rgb]{0.00,0.44,0.13}{\textbf{{#1}}}}
    \newcommand{\OperatorTok}[1]{\textcolor[rgb]{0.40,0.40,0.40}{{#1}}}
    \newcommand{\BuiltInTok}[1]{{#1}}
    \newcommand{\ExtensionTok}[1]{{#1}}
    \newcommand{\PreprocessorTok}[1]{\textcolor[rgb]{0.74,0.48,0.00}{{#1}}}
    \newcommand{\AttributeTok}[1]{\textcolor[rgb]{0.49,0.56,0.16}{{#1}}}
    \newcommand{\InformationTok}[1]{\textcolor[rgb]{0.38,0.63,0.69}{\textbf{\textit{{#1}}}}}
    \newcommand{\WarningTok}[1]{\textcolor[rgb]{0.38,0.63,0.69}{\textbf{\textit{{#1}}}}}
    
    
    % Define a nice break command that doesn't care if a line doesn't already
    % exist.
    \def\br{\hspace*{\fill} \\* }
    % Math Jax compatability definitions
    \def\gt{>}
    \def\lt{<}
    % Document parameters
    \title{Matlab OPP}
    
    
    

    % Pygments definitions
    
\makeatletter
\def\PY@reset{\let\PY@it=\relax \let\PY@bf=\relax%
    \let\PY@ul=\relax \let\PY@tc=\relax%
    \let\PY@bc=\relax \let\PY@ff=\relax}
\def\PY@tok#1{\csname PY@tok@#1\endcsname}
\def\PY@toks#1+{\ifx\relax#1\empty\else%
    \PY@tok{#1}\expandafter\PY@toks\fi}
\def\PY@do#1{\PY@bc{\PY@tc{\PY@ul{%
    \PY@it{\PY@bf{\PY@ff{#1}}}}}}}
\def\PY#1#2{\PY@reset\PY@toks#1+\relax+\PY@do{#2}}

\expandafter\def\csname PY@tok@w\endcsname{\def\PY@tc##1{\textcolor[rgb]{0.73,0.73,0.73}{##1}}}
\expandafter\def\csname PY@tok@c\endcsname{\let\PY@it=\textit\def\PY@tc##1{\textcolor[rgb]{0.25,0.50,0.50}{##1}}}
\expandafter\def\csname PY@tok@cp\endcsname{\def\PY@tc##1{\textcolor[rgb]{0.74,0.48,0.00}{##1}}}
\expandafter\def\csname PY@tok@k\endcsname{\let\PY@bf=\textbf\def\PY@tc##1{\textcolor[rgb]{0.00,0.50,0.00}{##1}}}
\expandafter\def\csname PY@tok@kp\endcsname{\def\PY@tc##1{\textcolor[rgb]{0.00,0.50,0.00}{##1}}}
\expandafter\def\csname PY@tok@kt\endcsname{\def\PY@tc##1{\textcolor[rgb]{0.69,0.00,0.25}{##1}}}
\expandafter\def\csname PY@tok@o\endcsname{\def\PY@tc##1{\textcolor[rgb]{0.40,0.40,0.40}{##1}}}
\expandafter\def\csname PY@tok@ow\endcsname{\let\PY@bf=\textbf\def\PY@tc##1{\textcolor[rgb]{0.67,0.13,1.00}{##1}}}
\expandafter\def\csname PY@tok@nb\endcsname{\def\PY@tc##1{\textcolor[rgb]{0.00,0.50,0.00}{##1}}}
\expandafter\def\csname PY@tok@nf\endcsname{\def\PY@tc##1{\textcolor[rgb]{0.00,0.00,1.00}{##1}}}
\expandafter\def\csname PY@tok@nc\endcsname{\let\PY@bf=\textbf\def\PY@tc##1{\textcolor[rgb]{0.00,0.00,1.00}{##1}}}
\expandafter\def\csname PY@tok@nn\endcsname{\let\PY@bf=\textbf\def\PY@tc##1{\textcolor[rgb]{0.00,0.00,1.00}{##1}}}
\expandafter\def\csname PY@tok@ne\endcsname{\let\PY@bf=\textbf\def\PY@tc##1{\textcolor[rgb]{0.82,0.25,0.23}{##1}}}
\expandafter\def\csname PY@tok@nv\endcsname{\def\PY@tc##1{\textcolor[rgb]{0.10,0.09,0.49}{##1}}}
\expandafter\def\csname PY@tok@no\endcsname{\def\PY@tc##1{\textcolor[rgb]{0.53,0.00,0.00}{##1}}}
\expandafter\def\csname PY@tok@nl\endcsname{\def\PY@tc##1{\textcolor[rgb]{0.63,0.63,0.00}{##1}}}
\expandafter\def\csname PY@tok@ni\endcsname{\let\PY@bf=\textbf\def\PY@tc##1{\textcolor[rgb]{0.60,0.60,0.60}{##1}}}
\expandafter\def\csname PY@tok@na\endcsname{\def\PY@tc##1{\textcolor[rgb]{0.49,0.56,0.16}{##1}}}
\expandafter\def\csname PY@tok@nt\endcsname{\let\PY@bf=\textbf\def\PY@tc##1{\textcolor[rgb]{0.00,0.50,0.00}{##1}}}
\expandafter\def\csname PY@tok@nd\endcsname{\def\PY@tc##1{\textcolor[rgb]{0.67,0.13,1.00}{##1}}}
\expandafter\def\csname PY@tok@s\endcsname{\def\PY@tc##1{\textcolor[rgb]{0.73,0.13,0.13}{##1}}}
\expandafter\def\csname PY@tok@sd\endcsname{\let\PY@it=\textit\def\PY@tc##1{\textcolor[rgb]{0.73,0.13,0.13}{##1}}}
\expandafter\def\csname PY@tok@si\endcsname{\let\PY@bf=\textbf\def\PY@tc##1{\textcolor[rgb]{0.73,0.40,0.53}{##1}}}
\expandafter\def\csname PY@tok@se\endcsname{\let\PY@bf=\textbf\def\PY@tc##1{\textcolor[rgb]{0.73,0.40,0.13}{##1}}}
\expandafter\def\csname PY@tok@sr\endcsname{\def\PY@tc##1{\textcolor[rgb]{0.73,0.40,0.53}{##1}}}
\expandafter\def\csname PY@tok@ss\endcsname{\def\PY@tc##1{\textcolor[rgb]{0.10,0.09,0.49}{##1}}}
\expandafter\def\csname PY@tok@sx\endcsname{\def\PY@tc##1{\textcolor[rgb]{0.00,0.50,0.00}{##1}}}
\expandafter\def\csname PY@tok@m\endcsname{\def\PY@tc##1{\textcolor[rgb]{0.40,0.40,0.40}{##1}}}
\expandafter\def\csname PY@tok@gh\endcsname{\let\PY@bf=\textbf\def\PY@tc##1{\textcolor[rgb]{0.00,0.00,0.50}{##1}}}
\expandafter\def\csname PY@tok@gu\endcsname{\let\PY@bf=\textbf\def\PY@tc##1{\textcolor[rgb]{0.50,0.00,0.50}{##1}}}
\expandafter\def\csname PY@tok@gd\endcsname{\def\PY@tc##1{\textcolor[rgb]{0.63,0.00,0.00}{##1}}}
\expandafter\def\csname PY@tok@gi\endcsname{\def\PY@tc##1{\textcolor[rgb]{0.00,0.63,0.00}{##1}}}
\expandafter\def\csname PY@tok@gr\endcsname{\def\PY@tc##1{\textcolor[rgb]{1.00,0.00,0.00}{##1}}}
\expandafter\def\csname PY@tok@ge\endcsname{\let\PY@it=\textit}
\expandafter\def\csname PY@tok@gs\endcsname{\let\PY@bf=\textbf}
\expandafter\def\csname PY@tok@gp\endcsname{\let\PY@bf=\textbf\def\PY@tc##1{\textcolor[rgb]{0.00,0.00,0.50}{##1}}}
\expandafter\def\csname PY@tok@go\endcsname{\def\PY@tc##1{\textcolor[rgb]{0.53,0.53,0.53}{##1}}}
\expandafter\def\csname PY@tok@gt\endcsname{\def\PY@tc##1{\textcolor[rgb]{0.00,0.27,0.87}{##1}}}
\expandafter\def\csname PY@tok@err\endcsname{\def\PY@bc##1{\setlength{\fboxsep}{0pt}\fcolorbox[rgb]{1.00,0.00,0.00}{1,1,1}{\strut ##1}}}
\expandafter\def\csname PY@tok@kc\endcsname{\let\PY@bf=\textbf\def\PY@tc##1{\textcolor[rgb]{0.00,0.50,0.00}{##1}}}
\expandafter\def\csname PY@tok@kd\endcsname{\let\PY@bf=\textbf\def\PY@tc##1{\textcolor[rgb]{0.00,0.50,0.00}{##1}}}
\expandafter\def\csname PY@tok@kn\endcsname{\let\PY@bf=\textbf\def\PY@tc##1{\textcolor[rgb]{0.00,0.50,0.00}{##1}}}
\expandafter\def\csname PY@tok@kr\endcsname{\let\PY@bf=\textbf\def\PY@tc##1{\textcolor[rgb]{0.00,0.50,0.00}{##1}}}
\expandafter\def\csname PY@tok@bp\endcsname{\def\PY@tc##1{\textcolor[rgb]{0.00,0.50,0.00}{##1}}}
\expandafter\def\csname PY@tok@fm\endcsname{\def\PY@tc##1{\textcolor[rgb]{0.00,0.00,1.00}{##1}}}
\expandafter\def\csname PY@tok@vc\endcsname{\def\PY@tc##1{\textcolor[rgb]{0.10,0.09,0.49}{##1}}}
\expandafter\def\csname PY@tok@vg\endcsname{\def\PY@tc##1{\textcolor[rgb]{0.10,0.09,0.49}{##1}}}
\expandafter\def\csname PY@tok@vi\endcsname{\def\PY@tc##1{\textcolor[rgb]{0.10,0.09,0.49}{##1}}}
\expandafter\def\csname PY@tok@vm\endcsname{\def\PY@tc##1{\textcolor[rgb]{0.10,0.09,0.49}{##1}}}
\expandafter\def\csname PY@tok@sa\endcsname{\def\PY@tc##1{\textcolor[rgb]{0.73,0.13,0.13}{##1}}}
\expandafter\def\csname PY@tok@sb\endcsname{\def\PY@tc##1{\textcolor[rgb]{0.73,0.13,0.13}{##1}}}
\expandafter\def\csname PY@tok@sc\endcsname{\def\PY@tc##1{\textcolor[rgb]{0.73,0.13,0.13}{##1}}}
\expandafter\def\csname PY@tok@dl\endcsname{\def\PY@tc##1{\textcolor[rgb]{0.73,0.13,0.13}{##1}}}
\expandafter\def\csname PY@tok@s2\endcsname{\def\PY@tc##1{\textcolor[rgb]{0.73,0.13,0.13}{##1}}}
\expandafter\def\csname PY@tok@sh\endcsname{\def\PY@tc##1{\textcolor[rgb]{0.73,0.13,0.13}{##1}}}
\expandafter\def\csname PY@tok@s1\endcsname{\def\PY@tc##1{\textcolor[rgb]{0.73,0.13,0.13}{##1}}}
\expandafter\def\csname PY@tok@mb\endcsname{\def\PY@tc##1{\textcolor[rgb]{0.40,0.40,0.40}{##1}}}
\expandafter\def\csname PY@tok@mf\endcsname{\def\PY@tc##1{\textcolor[rgb]{0.40,0.40,0.40}{##1}}}
\expandafter\def\csname PY@tok@mh\endcsname{\def\PY@tc##1{\textcolor[rgb]{0.40,0.40,0.40}{##1}}}
\expandafter\def\csname PY@tok@mi\endcsname{\def\PY@tc##1{\textcolor[rgb]{0.40,0.40,0.40}{##1}}}
\expandafter\def\csname PY@tok@il\endcsname{\def\PY@tc##1{\textcolor[rgb]{0.40,0.40,0.40}{##1}}}
\expandafter\def\csname PY@tok@mo\endcsname{\def\PY@tc##1{\textcolor[rgb]{0.40,0.40,0.40}{##1}}}
\expandafter\def\csname PY@tok@ch\endcsname{\let\PY@it=\textit\def\PY@tc##1{\textcolor[rgb]{0.25,0.50,0.50}{##1}}}
\expandafter\def\csname PY@tok@cm\endcsname{\let\PY@it=\textit\def\PY@tc##1{\textcolor[rgb]{0.25,0.50,0.50}{##1}}}
\expandafter\def\csname PY@tok@cpf\endcsname{\let\PY@it=\textit\def\PY@tc##1{\textcolor[rgb]{0.25,0.50,0.50}{##1}}}
\expandafter\def\csname PY@tok@c1\endcsname{\let\PY@it=\textit\def\PY@tc##1{\textcolor[rgb]{0.25,0.50,0.50}{##1}}}
\expandafter\def\csname PY@tok@cs\endcsname{\let\PY@it=\textit\def\PY@tc##1{\textcolor[rgb]{0.25,0.50,0.50}{##1}}}

\def\PYZbs{\char`\\}
\def\PYZus{\char`\_}
\def\PYZob{\char`\{}
\def\PYZcb{\char`\}}
\def\PYZca{\char`\^}
\def\PYZam{\char`\&}
\def\PYZlt{\char`\<}
\def\PYZgt{\char`\>}
\def\PYZsh{\char`\#}
\def\PYZpc{\char`\%}
\def\PYZdl{\char`\$}
\def\PYZhy{\char`\-}
\def\PYZsq{\char`\'}
\def\PYZdq{\char`\"}
\def\PYZti{\char`\~}
% for compatibility with earlier versions
\def\PYZat{@}
\def\PYZlb{[}
\def\PYZrb{]}
\makeatother


    % Exact colors from NB
    \definecolor{incolor}{rgb}{0.0, 0.0, 0.5}
    \definecolor{outcolor}{rgb}{0.545, 0.0, 0.0}



    
    % Prevent overflowing lines due to hard-to-break entities
    \sloppy 
    % Setup hyperref package
    \hypersetup{
      breaklinks=true,  % so long urls are correctly broken across lines
      colorlinks=true,
      urlcolor=urlcolor,
      linkcolor=linkcolor,
      citecolor=citecolor,
      }
    % Slightly bigger margins than the latex defaults
    
    \geometry{verbose,tmargin=1in,bmargin=1in,lmargin=1in,rmargin=1in}
    
    

    \begin{document}
    
    
    \maketitle
    
    

    
    \hypertarget{object-oriented-malab-programming}{%
\subsection{Object Oriented Malab
programming}\label{object-oriented-malab-programming}}

\hypertarget{objects}{%
\subsubsection{1. Objects}\label{objects}}

    \begin{Verbatim}[commandchars=\\\{\}]
{\color{incolor}In [{\color{incolor}13}]:} \PY{n}{a} \PY{p}{=} \PY{l+m+mi}{2}
         \PY{n}{class}\PY{p}{(}\PY{n}{a}\PY{p}{)}
\end{Verbatim}


    \begin{Verbatim}[commandchars=\\\{\}]

a =

     2


ans =

    'double'


    \end{Verbatim}

    \begin{Verbatim}[commandchars=\\\{\}]
{\color{incolor}In [{\color{incolor}14}]:} \PY{k}{methods}\PY{p}{(}\PY{n}{a}\PY{p}{)}
\end{Verbatim}


    \begin{Verbatim}[commandchars=\\\{\}]

Methods for class double:

abs               cscd              hess              polylog           
accumarray        csch              hypot             pow2              
acos              ctranspose        ichol             power             
acosd             cummax            igamma            psi               
acosh             cummin            ilu               qrupdate          
acot              cumprod           imag              rcond             
acotd             cumsum            inv               rdivide           
acoth             dawson            isbanded          real              
acsc              delete            isdiag            reallog           
acscd             det               isfinite          realpow           
acsch             diag              isinf             realsqrt          
airy              diff              isnan             rectangularPulse  
all               dilog             issorted          rem               
amd               dirac             issortedrows      reshape           
and               display           istril            round             
any               divisors          istriu            sec               
asec              dmperm            jacobiP           secd              
asecd             ei                jordan            sech              
asech             ellipticCE        kummerU           sign              
asin              ellipticCK        laguerreL         signIm            
asind             ellipticCPi       ldivide           sin               
asinh             ellipticE         le                sind              
atan              ellipticF         legendreP         sinh              
atan2             ellipticK         linsolve          sinhint           
atan2d            ellipticNome      log               sinint            
atand             ellipticPi        log10             sort              
atanh             eps               log1p             sortrowsc         
bernoulli         eq                log2              sparse            
besselh           erf               logint            sqrt              
besseli           erfc              lt                ssinint           
besselj           erfcinv           ltitr             superiorfloat     
besselk           erfcx             maxk              symrcm            
bessely           erfi              mink              tan               
betainc           erfinv            minpoly           tand              
betaincinv        euler             minus             tanh              
bsxfun            exp               mldivide          times             
ceil              expm1             mod               transpose         
charpoly          find              mpower            triangularPulse   
chebyshevT        fix               mrdivide          tril              
chebyshevU        floor             mtimes            triu              
colon             fresnelc          ne                uminus            
conj              fresnels          nnz               uplus             
cos               gamma             nonzeros          whittakerM        
cosd              gammainc          not               whittakerW        
cosh              gammaincinv       nzmax             wrightOmega       
coshint           gammaln           or                xor               
cosint            ge                ordeig            zeta              
cot               gegenbauerC       permute           
cotd              gt                plus              
coth              harmonic          pochhammer        
csc               hermiteH          poly2sym          


    \end{Verbatim}

    \begin{Verbatim}[commandchars=\\\{\}]
{\color{incolor}In [{\color{incolor}16}]:} \PY{n+nb}{log10}\PY{p}{(}\PY{n}{a}\PY{p}{)}
\end{Verbatim}


    \begin{Verbatim}[commandchars=\\\{\}]

ans =

    0.3010


    \end{Verbatim}

    \hypertarget{examples-from-alitrack}{%
\paragraph{Examples from Alitrack}\label{examples-from-alitrack}}

    \begin{Verbatim}[commandchars=\\\{\}]
{\color{incolor}In [{\color{incolor}5}]:} \PY{n}{p\PYZus{}folder} \PY{p}{=} \PY{l+s}{\PYZsq{}}\PY{l+s}{/Users/ryanlab/Desktop/AliT/Data/ALITracker\PYZus{}Data/aj031ro/aj031ro.edf\PYZsq{}}\PY{p}{;}
        \PY{n}{myparticipant} \PY{p}{=} \PY{n}{participant}\PY{p}{(}\PY{n}{p\PYZus{}folder}\PY{p}{,} \PY{l+s}{\PYZsq{}}\PY{l+s}{samples\PYZsq{}}\PY{p}{,} \PY{n}{true}\PY{p}{)}\PY{p}{;}
\end{Verbatim}


    \begin{Verbatim}[commandchars=\\\{\}]
Importing file /Users/ryanlab/Desktop/AliT/Data/ALITracker\_Data/aj031ro/aj031ro.edf{\ldots}
Loading 70 trials [|         ||        |||       ||||      |||||     ||||||    |||||||   ||||||||  ||||||||| ||||||||||]
Truncating samples to exclude empty ones
done!

    \end{Verbatim}

    \begin{Verbatim}[commandchars=\\\{\}]
{\color{incolor}In [{\color{incolor}6}]:} \PY{c}{\PYZpc{}Object orieneted easy to read code, easy to us}
        \PY{n}{myparticipant}\PY{p}{.}\PY{n}{set\PYZus{}trial\PYZus{}features}\PY{p}{(}\PY{l+m+mi}{1}\PY{p}{:}\PY{l+m+mi}{70}\PY{p}{,}\PY{l+s}{\PYZsq{}}\PY{l+s}{start\PYZus{}event\PYZsq{}}\PY{p}{,}\PYZdq{}\PY{n}{Study\PYZus{}display}\PYZdq{}\PY{p}{,} \PY{l+s}{\PYZsq{}}\PY{l+s}{end\PYZus{}event\PYZsq{}}\PY{p}{,} \PYZdq{}\PY{n}{Study\PYZus{}timer}\PYZdq{}\PY{p}{)}
\end{Verbatim}


    \begin{Verbatim}[commandchars=\\\{\}]
{\color{incolor}In [{\color{incolor}7}]:} \PY{c}{\PYZpc{}Making a new object. }
        \PY{n}{new\PYZus{}trial} \PY{p}{=} \PY{n}{myparticipant}\PY{p}{.}\PY{n}{gettrial}\PY{p}{(}\PY{l+m+mi}{1}\PY{p}{)}
\end{Verbatim}


    \begin{Verbatim}[commandchars=\\\{\}]

new\_trial = 

  trial with properties:

                 parent: [1x1 participant]
                   data: []
               trial\_no: 1
        trial\_fieldname: 'trial\_1'
            num\_samples: 3992
                  index: [1x3992 uint32]
            sample\_time: [1x3992 uint32]
             trial\_time: [1x3992 uint32]
                      x: [1x3992 double]
                      y: [1x3992 double]
                    rho: []
                  theta: []
    issaccadeorfixation: []
              fixations: [1x1 struct]
             isfixation: []
               saccades: [1x1 struct]
              issaccade: []
                   rois: [1x1 struct]
              condition: []


    \end{Verbatim}

    \begin{Verbatim}[commandchars=\\\{\}]
{\color{incolor}In [{\color{incolor}8}]:} \PY{n}{new\PYZus{}trial}\PY{p}{.}\PY{n}{saccades}
\end{Verbatim}


    \begin{Verbatim}[commandchars=\\\{\}]

ans = 

  struct with no fields.


    \end{Verbatim}

    \begin{Verbatim}[commandchars=\\\{\}]
{\color{incolor}In [{\color{incolor}9}]:} \PY{n}{new\PYZus{}trial}\PY{p}{.}\PY{n}{number\PYZus{}of\PYZus{}saccade}\PY{p}{(}\PY{p}{)}
        \PY{n}{new\PYZus{}trial}\PY{p}{.}\PY{n}{saccades}
\end{Verbatim}


    \begin{Verbatim}[commandchars=\\\{\}]

ans = 

  struct with fields:

    rawindex: [3x1 double]
      number: 3
       start: [736000 1869000 2597000]
         end: [744000 1876000 2615000]


    \end{Verbatim}

    \begin{Verbatim}[commandchars=\\\{\}]
{\color{incolor}In [{\color{incolor}10}]:} \PY{n}{new\PYZus{}trial}\PY{p}{.}\PY{n}{number\PYZus{}of\PYZus{}fixation}\PY{p}{(}\PY{p}{)}
         \PY{n}{new\PYZus{}trial}\PY{p}{.}\PY{n}{number\PYZus{}of\PYZus{}saccade}\PY{p}{(}\PY{p}{)}
         \PY{n}{new\PYZus{}trial}\PY{p}{.}\PY{n}{duration\PYZus{}of\PYZus{}fixation}\PY{p}{(}\PY{p}{)}
         \PY{n}{new\PYZus{}trial}\PY{p}{.}\PY{n}{duration\PYZus{}of\PYZus{}saccade}\PY{p}{(}\PY{p}{)}
         \PY{n}{new\PYZus{}trial}\PY{p}{.}\PY{n}{location\PYZus{}of\PYZus{}fixation}\PY{p}{(}\PY{p}{)}
         \PY{n}{new\PYZus{}trial}\PY{p}{.}\PY{n}{location\PYZus{}of\PYZus{}saccade}\PY{p}{(}\PY{p}{)}
         \PY{n}{new\PYZus{}trial}\PY{p}{.}\PY{n}{amplitude\PYZus{}of\PYZus{}saccade}\PY{p}{(}\PY{p}{)}
         \PY{n}{new\PYZus{}trial}\PY{p}{.}\PY{n}{deviation\PYZus{}of\PYZus{}duration\PYZus{}of\PYZus{}fixation}
         \PY{n}{new\PYZus{}trial}\PY{p}{.}\PY{n}{deviation\PYZus{}of\PYZus{}duration\PYZus{}of\PYZus{}saccade}
         \PY{n}{new\PYZus{}trial}\PY{p}{.}\PY{n}{get\PYZus{}polar}
         \PY{n}{new\PYZus{}trial}\PY{p}{.}\PY{n}{get\PYZus{}issaccade}
         \PY{n}{new\PYZus{}trial}\PY{p}{.}\PY{n}{get\PYZus{}isfixation}
         
         \PY{n}{new\PYZus{}trial}
\end{Verbatim}


    \begin{Verbatim}[commandchars=\\\{\}]

new\_trial = 

  trial with properties:

                 parent: [1x1 participant]
                   data: []
               trial\_no: 1
        trial\_fieldname: 'trial\_1'
            num\_samples: 3992
                  index: [1x3992 uint32]
            sample\_time: [1x3992 uint32]
             trial\_time: [1x3992 uint32]
                      x: [1x3992 double]
                      y: [1x3992 double]
                    rho: [1x3992 double]
                  theta: [1x3992 double]
    issaccadeorfixation: []
              fixations: [1x1 struct]
             isfixation: [1x3992 double]
               saccades: [1x1 struct]
              issaccade: [1x3992 double]
                   rois: [1x1 struct]
              condition: []


    \end{Verbatim}

    \begin{Verbatim}[commandchars=\\\{\}]
{\color{incolor}In [{\color{incolor}11}]:} \PY{n}{new\PYZus{}trial}\PY{p}{.}\PY{n}{saccades}
\end{Verbatim}


    \begin{Verbatim}[commandchars=\\\{\}]

ans = 

  struct with fields:

              rawindex: [3x1 double]
                number: 3
                 start: [736000 1869000 2597000]
                   end: [744000 1876000 2615000]
              duration: [1123000 719000 1375000]
           start\_gazex: [3x1 double]
           start\_gazey: [3x1 double]
             end\_gazex: [3x1 double]
             end\_gazey: [3x1 double]
             amplitude: [3x1 double]
    duration\_variation: [0.1531 -1.0677 0.9146]


    \end{Verbatim}

    \begin{Verbatim}[commandchars=\\\{\}]
{\color{incolor}In [{\color{incolor}12}]:} \PY{n}{new\PYZus{}trial}\PY{p}{.}\PY{n}{animate}
\end{Verbatim}


    \begin{center}
    \adjustimage{max size={0.9\linewidth}{0.9\paperheight}}{output_12_0.png}
    \end{center}
    { \hspace*{\fill} \\}
    
    \hypertarget{creating-classes}{%
\subsubsection{2. Creating Classes}\label{creating-classes}}

    \hypertarget{excerpt-from-matlabs-website}{%
\paragraph{excerpt from matlab's
website}\label{excerpt-from-matlabs-website}}

MATLAB classes use the following words to describe different parts of a
class definition and related concepts.

Class definition --- Description of what is common to every instance of
a class.

\hypertarget{properties-data-storage-for-class-instances}{%
\subparagraph{Properties --- Data storage for class
instances}\label{properties-data-storage-for-class-instances}}

\hypertarget{methods-special-functions-that-implement-operations-that-are-usually-performed-only-on-instances-of-the-class}{%
\subparagraph{Methods --- Special functions that implement operations
that are usually performed only on instances of the
class}\label{methods-special-functions-that-implement-operations-that-are-usually-performed-only-on-instances-of-the-class}}

    \hypertarget{events-messages-defined-by-classes-and-broadcast-by-class-instances-when-some-specific-action-occurs}{%
\subparagraph{Events --- Messages defined by classes and broadcast by
class instances when some specific action
occurs}\label{events-messages-defined-by-classes-and-broadcast-by-class-instances-when-some-specific-action-occurs}}

\hypertarget{attributes-values-that-modify-the-behavior-of-properties-methods-events-and-classes}{%
\subparagraph{Attributes --- Values that modify the behavior of
properties, methods, events, and
classes}\label{attributes-values-that-modify-the-behavior-of-properties-methods-events-and-classes}}

\hypertarget{listeners-objects-that-respond-to-a-specific-event-by-executing-a-callback-function-when-the-event-notice-is-broadcast}{%
\subparagraph{Listeners --- Objects that respond to a specific event by
executing a callback function when the event notice is
broadcast}\label{listeners-objects-that-respond-to-a-specific-event-by-executing-a-callback-function-when-the-event-notice-is-broadcast}}

\hypertarget{objects-instances-of-classes-which-contain-actual-data-values-stored-in-the-objects-properties}{%
\subparagraph{Objects --- Instances of classes, which contain actual
data values stored in the objects'
properties}\label{objects-instances-of-classes-which-contain-actual-data-values-stored-in-the-objects-properties}}

    \begin{Verbatim}[commandchars=\\\{\}]
{\color{incolor}In [{\color{incolor}19}]:} \PY{n}{imshow}\PY{p}{(}\PY{l+s}{\PYZsq{}}\PY{l+s}{1.object\PYZus{}oriented/assets/classes\PYZus{}matlab.png\PYZsq{}}\PY{p}{)}
\end{Verbatim}


    \begin{center}
    \adjustimage{max size={0.9\linewidth}{0.9\paperheight}}{output_16_0.png}
    \end{center}
    { \hspace*{\fill} \\}
    
    \hypertarget{subclasses-classes-that-are-derived-from-other-classes-and-that-inherit-the-methods-properties-and-events-from-those-classes-subclasses-facilitate-the-reuse-of-code-defined-in-the-superclass-from-which-they-are-derived.}{%
\subparagraph{Subclasses --- Classes that are derived from other classes
and that inherit the methods, properties, and events from those classes
(subclasses facilitate the reuse of code defined in the superclass from
which they are
derived).}\label{subclasses-classes-that-are-derived-from-other-classes-and-that-inherit-the-methods-properties-and-events-from-those-classes-subclasses-facilitate-the-reuse-of-code-defined-in-the-superclass-from-which-they-are-derived.}}

\hypertarget{superclasses-classes-that-are-used-as-a-basis-for-the-creation-of-more-specifically-defined-classes-that-is-subclasses.}{%
\subparagraph{Superclasses --- Classes that are used as a basis for the
creation of more specifically defined classes (that is,
subclasses).}\label{superclasses-classes-that-are-used-as-a-basis-for-the-creation-of-more-specifically-defined-classes-that-is-subclasses.}}

\hypertarget{packages-folders-that-define-a-scope-for-class-and-function-naming}{%
\subparagraph{Packages --- Folders that define a scope for class and
function
naming}\label{packages-folders-that-define-a-scope-for-class-and-function-naming}}

    \begin{Verbatim}[commandchars=\\\{\}]
{\color{incolor}In [{\color{incolor} }]:} \PY{n}{myparticipant}
\end{Verbatim}


    \begin{Verbatim}[commandchars=\\\{\}]

myparticipant = 

  participant with properties:

             id: []
        address: []
         trials: \{1x70 cell\}
           subs: \{[31]\}
           data: \{[1x70 struct]\}
            raw: []
    subject\_var: 'ID'
         screen: [1x1 struct]
           rois: [1x1 struct]
            log: [1x1 struct]


    \end{Verbatim}

    \hypertarget{handles}{%
\subsubsection{3. Handles}\label{handles}}

    \begin{Verbatim}[commandchars=\\\{\}]
{\color{incolor}In [{\color{incolor}24}]:} \PY{c}{\PYZpc{} Example directory from matlab\PYZsq{}s website }
         
         \PY{c}{\PYZpc{}function myFunc(var)}
         \PY{c}{\PYZpc{}   var = var + 1;}
         \PY{c}{\PYZpc{}end}
         
         \PY{n}{myFunc} \PY{p}{=} \PY{p}{@}\PY{p}{(}\PY{n}{x}\PY{p}{)} \PY{n}{x}\PY{o}{+}\PY{l+m+mi}{1}\PY{p}{;}
         
         \PY{n}{x} \PY{p}{=} \PY{l+m+mi}{12}\PY{p}{;}
         \PY{n}{myFunc}\PY{p}{(}\PY{n}{x}\PY{p}{)}\PY{p}{;}
         \PY{n}{x}
\end{Verbatim}


    \begin{Verbatim}[commandchars=\\\{\}]

x =

    12


    \end{Verbatim}

    \hypertarget{resources}{%
\subsubsection{3. Resources}\label{resources}}

\hypertarget{external-relevant-links}{%
\paragraph{External relevant links}\label{external-relevant-links}}

\begin{itemize}
\tightlist
\item
  \href{https://ocw.mit.edu/courses/electrical-engineering-and-computer-science/6-0001-introduction-to-computer-science-and-programming-in-python-fall-2016/lecture-slides-code/MIT6_0001F16_Lec8.pdf}{Referenced
  mit lecture slides}
\item
  \href{https://ocw.mit.edu/courses/electrical-engineering-and-computer-science/6-0001-introduction-to-computer-science-and-programming-in-python-fall-2016/lecture-videos/lecture-8-object-oriented-programming/}{Reference
  mit lecture video}
\item
  \href{https://www.mathworks.com/help/matlab/object-oriented-design-with-matlab.html}{Matlab
  documentation}
\end{itemize}

\hypertarget{my-personal-links}{%
\paragraph{My personal links}\label{my-personal-links}}

\begin{itemize}
\tightlist
\item
  \href{https://codepen.io/ATajadod94/pen/pKvOLE?editors=0010}{Javascript
  javascript code}
\item
  \href{https://github.com/ATajadod94/ALITrack/tree/master/src}{Referenced
  matlab code}
\item
  \href{https://github.com/ATajadod94/Python_tutorial/tree/master/Code/Object_oriented}{this
  talk's directory on github}
\end{itemize}

    written by Alireza Tajdod


    % Add a bibliography block to the postdoc
    
    
    
    \end{document}
